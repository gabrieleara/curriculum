%!TEX TS-program = xelatex

\PassOptionsToPackage{inline}{enumitem}

% Options:
% - online: uses "fake" letters for @ and . to avoid spam when posting the
%   resume on a webpage
% - print: prints more information (address, phone number, etc) that you may not
%   put in an online version of this document (also uses less color)
% - extended: use if you have stuff within \ifextended environments, to print
%   out more information when you want to
% - colorful: forces the use of colors also for the print version
% - en/it/fr/de/etc: determine the language of this document
\documentclass[a4paper,en,print,online]{adcv-template/adcv}

\usepackage[english,italian]{babel}
\usepackage{nth}
\usepackage{xspace}

\newcommand{\positionheading}[4]{
\ifextended
  \adcvrowtwo{\textbf{#2}}{#1}
  \adcvrowtwo{#3}{}
  \ifx#4\empty\else
  \adcvrowtwo{#4}{}
  \fi
\else
  \adcvrowtwo{\textbf{#2}, #3}{#1}
\fi
}

\newcommand{\positionheadinglong}[4]{
\ifextended
  \adcvrowtwo{\textbf{#2}}{#1}
  \adcvrowtwo{#3}{}
  \ifx#4\empty\else
  \adcvrowtwo{#4}{}
  \fi
\else
  \adcvrowtwo{\textbf{#2}}{#1}
  \adcvrowtwo{#3}{}
\fi
}

\newcommand{\unipi}{University of Pisa\xspace}
\newcommand{\unipisa}{\unipi, Pisa (Italy)\xspace}

\newcommand{\santanna}{Scuola Superiore Sant'Anna\xspace}
\newcommand{\santannapisa}{\santanna, Pisa (Italy)\xspace}

\newcommand{\phd}{Ph.D.\@\xspace}
\newcommand{\philod}{Philosophiae Doctor (\phd)\xspace}

\newcommand{\msc}{M.Sc.\@\xspace}
\newcommand{\mastersc}{Master of Science (\msc)\xspace}

\newcommand{\bsc}{B.Sc.\@\xspace}
\newcommand{\bachelorsc}{Bachelor of Science (\bsc)\xspace}

\newcommand{\tomcucinotta}{\href{https://retis.santannapisa.it/~tommaso}{Prof.\ Tommaso Cucinotta}\xspace}


\title{Gabriele Ara’s CV}

\adcvname{Gabriele}{Ara}{Eng}
\adcvtitle{Embedded Systems, Real-Time Systems, System Programming, Research}
\adcvaddress{Piazza Due Giugno 15}{57122}{Livorno}{Italy}
\adcvwebsite{https://www.gabrieleara.it}{www.gabrieleara.it}
\adcvemail{gabriele.ara}{santannapisa}{it}
\adcvphone{(+39) 338 419 1704}
\adcvdate{October 2021}

\addbibresource{biblio.bib}

\begin{document}

Gabriele Ara is a \phd Student at \santannapisa. He has more than 10 years of experience in system programming.


% \ifextended
  Gabriele’s expertise spans multiple fields in Computer and Software
  Engineering and IT, including embedded systems, real-time systems, scheduling,
  computer architectures, software design and implementation, system
  programming, networking, and research.

  He is currently a proud member of the Real-Time Systems Lab at \santanna under
  the supervision of \tomcucinotta, and his research interests include
  energy-aware scheduling of real-time systems on embedded platforms running
  Linux, high-performance network communications in cloud environments, and the
  simulation of multi-core real-time systems.

  Gabriele has a \bsc in Computer Engineering (cum laude) from the \unipisa, and
  an \msc in Embedded Computing Systems Engineering (cum laude), a course
  jointly offered by the \unipi and \santannapisa.
% \fi

Gabriele speaks fluently Italian and English (almost) alike. His colleagues
describe him as analytical, creative, competitive, and goal-oriented.

\section{Current Position}
\begin{adcvtabletwo}
  \positionheading
    {2019--Present}
    {\phd Student}
    {\santannapisa}
    {\url{www.santannapisa.it}}
  \adcvrowtwo{Supervisor: \tomcucinotta}{}
  \adcvrowmulti{Research topics include
    \begin{itemize*}[afterlabel= , label=]
      \item Energy-Aware Scheduling of Real-Time Tasks,
      \item Dynamic Voltage and Frequency Scaling (DVFS) on Embedded Platforms,
      \item Real-Time Systems Simulation and Scheduling, and
      \item High-Performance Networking Stacks and Frameworks.
    \end{itemize*}
  }
\end{adcvtabletwo}

\section{Experience}\label{sec:experience}

\begin{adcvtabletwo}
  \positionheading
    {2013--Present}
    {Textbook Author}
    {Zanichelli editore S.p.A (Italy)}
    {\url{www.zanichelli.it}}
  %
  \adcvrowmulti{Gabriele writes from time to time chapters and sections of IT
    textbooks in Italian. The chapters focus on mobile application development
    essentials for Android OS, from the very basics to more advanced use-case
    examples. The complete list of textbooks to which Gabriele collaborated is
    in \hyperref[sec:bookchapters]{Book Chapters}.}
  \adcvrowskip

  %----------------------------------------------------------------------------%

  \positionheadinglong
    {2018--2019}
    {High School Teaching Professional}
    {Istituto di Istruzione Superiore "Vespucci-Colombo", Livorno (Italy)}
    {\url{vespucci.edu.it}}
  %
  \adcvrowmulti{For one academic year, Gabriele worked as a part-time professor
    in an Italian high school, teaching IT and IT Laboratory to both \nth{10}
    and \nth{11}-grade students.}
  \adcvrowskip

  %----------------------------------------------------------------------------%

  \positionheading
    {2017--2018}
    {E-learning Technical Specialist}
    {\unipisa}
    {\url{www.unipi.it}}
  %
  \adcvrowmulti{To support my studies, I worked a couple of years as on-site
    support to technical and teaching personnel for the University for practical
    aspects of live streaming classes and conferences and the post-production of
    videos of an e-learning platform. Duties included managing recording
    equipment, streaming software, and providing hands-on support in case of
    live failures.}
  \adcvrowskip

  %----------------------------------------------------------------------------%

  \positionheading
    {2013--2018}
    {Tutor}
    {Independent Contractor, Livorno (Italy)}
    {}
  %
  \adcvrowmulti{Throughout my university years, I managed to help some high
    school students in STEM fields as their tutor, helping them with assignments
    in classes such as Computer Engineering, IT, Maths, Electronics, and
    Physics.}

  %----------------------------------------------------------------------------%

\end{adcvtabletwo}

%------------------------------------------------------------------------------%
%------------------------------------------------------------------------------%

\ifextended
\clearpage
\else
\fi

\section{Awards}\label{sec:awards}
\begin{adcvtabletwo}
  \adcvrowtwo{\textbf{Best Paper Award}}{2020} \adcvrowmulti{Gabriele Ara,
  Tommaso Cucinotta, Luca Abeni, Carlo Vitucci, for the work
  "\textit{\textbf{Comparative Evaluation of Kernel Bypass Mechanisms for
  High-performance Inter-container Communications}}" at the 10th International
  Conference on Cloud Computing and Services Science (CLOSER), 2020}
\end{adcvtabletwo}

%------------------------------------------------------------------------------%
%------------------------------------------------------------------------------%

\ifextended
\else
\clearpage
\fi
\section{Education}\label{sec:education}

\begin{adcvtabletwo}
  \adcvrowtwo{\textbf{\mastersc} in Embedded Computing Systems}{2019}
  \adcvrowtwo{\unipi{} {\textbf{and}} \santannapisa}{}
  %
  \ifextended
    \adcvrowmulti{
      Specialized curriculum in system programming, embedded and real-time
      systems, mechatronics, computer architectures and component frameworks.
      %
      Final grade: \textit{Summa cum laude}. Subject of the dissertation: {\em
      Design and Implementation of a Performance Testing Framework for
      High-Performance Inter-Container Communications}. Supervisor: {\em
      \tomcucinotta}.
    }
  \fi
  \adcvrowskip

  %----------------------------------------------------------------------------%

  \adcvrowtwo{\textbf{\bachelorsc} in Computer Engineering}{2016}
  \adcvrowtwo{\unipisa}{}
  %
  \ifextended
    \adcvrowmulti{
      Specialized curriculum in computer science and engineering, computer
      architectures, system programming, industrial automation and control
      systems.
      %
      Final grade: \textit{Summa cum laude}. Subject of the dissertation (in
      Italian): {\em Dynamic and Interactive Crisis Mapping and Generation}.
      Supervisor: {\em Prof.\ Marco Avvenuti}.
    }
  \fi
  \adcvrowskip

  %----------------------------------------------------------------------------%

  \adcvrowtwo{\textbf{Secondary Diploma} as IT Professional}{2013}

  \adcvrowtwo{Istituto Tecnico Industriale Statale G.\ Galilei, Livorno (Italy)}{}
  \ifextended
  \adcvrowtwo{\url{www.galileilivorno.gov.it}}{}
  \fi
  %
  \ifextended
    \adcvrowmulti{
      Curriculum in computer science, programming patterns, web technologies,
      mobile app development, networking, and system programming.
      %
      Final grade: \textit{100/100}.
    }
  \fi

  %----------------------------------------------------------------------------%

\end{adcvtabletwo}

%------------------------------------------------------------------------------%
%------------------------------------------------------------------------------%

\section{Supervision}\label{sec:supervision}
\begin{adcvtabletwo}
  \adcvrowtwo{\textbf{Leonardo Lai}}{2019}
  \ifextended
    \adcvrowtwo{Master student in Embedded Computing Systems}{}
    \adcvrowtwo{\santannapisa}{}
  \else
  \adcvrowtwo{Master student in Embedded Computing Systems, \santannapisa}{}
  \fi
  %
  \adcvrowmulti{Subject of the dissertation: {\em Implementation and Evaluation
  of High-Performance Userspace Networking Mechanisms for Virtualized Network
  Functions}}

  %----------------------------------------------------------------------------%
\end{adcvtabletwo}

%------------------------------------------------------------------------------%
%------------------------------------------------------------------------------%

\section{Professional Skills}\label{sec:skills}

\newcommand*{\lang}[1]{{\linktext \textbf{#1}}\xspace}
% \multicolumn{2}{@{}p{\textwidth}}{\lighttext #1}\tabularnewline

\begin{adcvtabletwo}
  \adcvrowmultifake{\textbf{Communication skills} Comfortable interacting with
    people with cross-cultural backgrounds from around the world. Gave
    presentations to moderate international audiences for conferences and
    workshops. He can communicate strengths and weaknesses of his work.}
  \adcvrowskip

  %----------------------------------------------------------------------------%

  \adcvrowmultifake{\textbf{Organisational/Managerial skills} He can both work
    in small teams and manage to get things done by himself. Comfortable leading
    other people when necessary to organize the work. Excellent debugging and
    investigative skills for software/computer systems.}
  \adcvrowskip

  %----------------------------------------------------------------------------%

  \adcvrowskip
  \adcvrowtwo{\large \textbf{Job-related skills}}{}
  \adcvrowskip
  \adcvrowmulti{Following is the list of known programming languages and tools
  organized by proficiency level.}
  \adcvrowskip

  \adcvrowtwo{\textbf{Excellent knowledge and proficiency:}}{}
  \adcvrowmulti{
    \begin{itemize}
      \item \lang{C/C++}, including the understanding of language internals,
      standard libraries, POSIX, Linux-specific libraries, Linux system
      programming, microcontrollers programming.
    \end{itemize}
  }

  \adcvrowtwo{\textbf{Proficient knowledge:}}{}
  \adcvrowmulti{
    \begin{itemize}
      \item \lang{Virtualization} technologies, including Docker, LXC, LXD, and
      custom solutions based on {\em cgroups} and {\em namespaces};
      \item \lang{High-Performance Networking} frameworks, including DPDK, FD.io
      VPP, Open vSwitch;
      \item \lang{Linux} management and administrative tools for machine
        configuration and troubleshooting.
      \item \lang{Unix}-based systems command line, scripting in \lang{Sh} and
        \lang{Bash};
      \item \lang{Java} (standard libraries, JavaFX, Spring, Apache and Servlet
        libraries);
      \item \lang{Android} App programming (using Java+XML);
    \end{itemize}
  }
  % \adcvrowskip

  \adcvrowtwo{\textbf{Good knowledge:}}{}
  \adcvrowmulti{
    \begin{itemize}
      \item \lang{Linux kernel} internals and hacking;
      \item \lang{JavaScript} (standard libraries, JQuery, D3);
      \item \lang{Matlab} and Simulink for scripting, data analysis, and system
        control or simulation;
      \item \lang{Web} development using HTML5, PHP5, CSS3, XML, JSON, REST
        patterns;
      \item \lang{SQL} and MySQL software suite;
      \item \lang{CISCO} technologies and CISCO IOS;
    \end{itemize}
  }
  % \adcvrowskip

  %----------------------------------------------------------------------------%
\end{adcvtabletwo}

%------------------------------------------------------------------------------%
%------------------------------------------------------------------------------%

\section{Languages}\label{sec:languages}

\ifextended
  \begin{adcvlanguages}
    \adcvmothertongue{Italian}
    \adcvlanguagesheader
    \adcvlanguage{2}{English}{\adcvCTwo}{\adcvCTwo}{\adcvCTwo}{\adcvCOne}{\adcvCTwo}
    \adcvlanguagesfooter
    \adcvlanguagesfootnote{2}{2012 -- Trinity's Graded Examinations in Spoken English - Grade 7 (CEFR B2)}
  \end{adcvlanguages}
\else
  \textbf{Italian}: Native proficiency
  \textbf{English}: Full professional proficiency
\fi

%------------------------------------------------------------------------------%
%------------------------------------------------------------------------------%

\section{Personal Interests}\label{sec:interests}

Science, Technology, Traveling, Playing and Listening to Music, Food, Movies,
Trekking

%------------------------------------------------------------------------------%
%------------------------------------------------------------------------------%

\ifextended
\vspace{3em}
% \clearpage
\else
\ifpublist
\clearpage
\fi
\fi
\section{Publications}\label{sec:publications}

\newcommand{\fullpublicationslist}{
  \begin{refsection}
    \nocite{Ara20CCIS}
    \nocite{Serra21JSA}
    \printbibliography[title={Peer-reviewed Journals\label{sec:journals}}, heading=subbibliography]
  \end{refsection}

  \begin{refsection}
    \nocite{Serra20}
    \nocite{Ara20CLOSER}
    \nocite{Ara19}
    \printbibliography[title={Peer-reviewed Conference and Workshop Proceedings\label{sec:conferences}}, heading=subbibliography]
  \end{refsection}

  \ifextended
    \clearpage
  \fi
  \begin{refsection}
    \nocite{MC2020}
    \nocite{MF2017}
    \nocite{MC2015}
    \nocite{MF2014}
    \printbibliography[title={Book Chapters\label{sec:bookchapters}}, heading=subbibliography]
  \end{refsection}
}

\ifextended
  \fullpublicationslist
\else
  \ifpublist
    \fullpublicationslist
  \else
    Search me on
    \href{htts://scholar.google.com/citations?user=nl1RmecAAAAJ}{Google Scholar}
    for the list of my peer-reviewed conference proceedings and journals. Or
    have a look at \href{https://gabrieleara.github.io/publications}{my
    publications page}.
  \fi
\fi


%------------------------------------------------------------------------------%
%------------------------------------------------------------------------------%

\ifextended
\section{Presentations}\label{sec:presentations}

\begin{adcvpresentations}
  \adcvpresentation{On the Use of Kernel Bypass Mechanisms for High-Performance
    Inter-container Communications}{Frankfurt, Germany, June 20, 2019}

  \adcvpresentation{\href{https://www.youtube.com/watch?v=cQ3ecv6TVZc}{Comparative
    Evaluation of Kernel Bypass Mechanisms for High-performance Inter-container
    Communications}}{[Online] Prague, Czech Republic, May 7, 2020}
\end{adcvpresentations}
\fi

\end{document}
