%!TEX TS-program = xelatex

\PassOptionsToPackage{inline}{enumitem}

% Options:
% - online: uses "fake" letters for @ and . to avoid spam when posting the
%   resume on a webpage
% - print: prints more information (address, phone number, etc) that you may not
%   put in an online version of this document (also uses less color)
% - extended: use if you have stuff within % \ifextended environments, to print
%   out more information when you want to
% - colorlinks: colorize links
% - colorbar: colorize top bar on first page
% - en/it/fr/de/etc: determine the language of this document
\documentclass[a4paper,en,print,colorbar,colorlinks,extended]{adcv-template/adcv}

\usepackage[english]{babel}
\usepackage{nth}
\usepackage{xspace}

\newcommand{\positionheading}[4]{
\ifextended
  \adcvrowtwo{\textbf{#2}}{#1}
  \adcvrowtwo{#3}{}
  \ifx#4\empty\else
  \adcvrowtwo{#4}{}
  \fi
\else
  \adcvrowtwo{\textbf{#2}, #3}{#1}
\fi
}

\newcommand{\positionheadinglong}[4]{
\ifextended
  \adcvrowtwo{\textbf{#2}}{#1}
  \adcvrowtwo{#3}{}
  \ifx#4\empty\else
  \adcvrowtwo{#4}{}
  \fi
\else
  \adcvrowtwo{\textbf{#2}}{#1}
  \adcvrowtwo{#3}{}
\fi
}

\newcommand{\unipi}{University of Pisa\xspace}
\newcommand{\unipisa}{\unipi, Pisa (Italy)\xspace}

\newcommand{\santanna}{Scuola Superiore Sant'Anna\xspace}
\newcommand{\santannapisa}{\santanna, Pisa (Italy)\xspace}

\newcommand{\phd}{Ph.D.\@\xspace}
\newcommand{\philod}{Philosophiae Doctor (\phd)\xspace}

\newcommand{\msc}{M.Sc.\@\xspace}
\newcommand{\mastersc}{Master of Science (\msc)\xspace}

\newcommand{\bsc}{B.Sc.\@\xspace}
\newcommand{\bachelorsc}{Bachelor of Science (\bsc)\xspace}

\newcommand{\tomcucinotta}{\href{https://retis.santannapisa.it/~tommaso}{Prof.\ Tommaso Cucinotta}\xspace}
\newcommand{\lucabenini}{\href{https://ee.ethz.ch/the-department/people-a-z/person-detail.luca-benini.html}{Prof.\ Luca Benini}\xspace}

\title{Gabriele Ara's CV}

\adcvname{Gabriele}{Ara}{Eng}
\adcvtitle{Embedded Systems, Real-Time Systems, System Programming, Research}
\adcvaddress{Piazza Due Giugno 15}{57122}{Livorno}{Italy}
\adcvwebsite{https://www.gabrieleara.it}{www.gabrieleara.it}
\adcvemail{gabriele.ara}{santannapisa}{it}
\adcvphone{(+39) 338 419 1704}
\adcvdate{October 2022}

\addbibresource{biblio.bib}

\begin{document}

I am a \phd Student at Scuola Superiore Sant'Anna, Pisa (Italy), and I have
more than 10 years of experience in system programming.

My expertise spans multiple fields in Computer and Software Engineering and IT,
including embedded systems, real-time systems scheduling, computer
architectures, software design and implementation, system programming,
networking, kernel development, mobile app development, and research.

My research interests include energy-aware scheduling of real-time systems on
embedded platforms running Linux, high-performance and low latency network
communications in cloud environments, and the simulation of multi-core real-time
systems.

I have an M.Sc. in Embedded Computing Systems Engineering (cum laude), a course
jointly offered by the Scuola Superiore Sant'Anna of Pisa and the University of
Pisa (Italy), and a B.Sc. in Computer Engineering (cum laude) from the
University of Pisa.

I fluently speak Italian and English alike. My colleagues often describe me as
analytical, creative, competitive, and goal-oriented.

\section{Current Position}
\begin{adcvtabletwo}
  \positionheading
    {2019--Present}
    {International \phd Student}
    {\santannapisa}
    {\url{www.santannapisa.it}}
  \adcvrowtwo{Supervisor: \tomcucinotta}{}
  \adcvrowmulti{Research topics include
    \begin{itemize*}[afterlabel= , label=]
      \item Energy-Aware Scheduling of Real-Time Tasks,
      \item Dynamic Voltage and Frequency Scaling (DVFS) on Embedded Platforms,
      \item Real-Time Systems Simulation and Scheduling, and
      \item High-Performance Networking Stacks and Frameworks.
    \end{itemize*}
  }
\end{adcvtabletwo}

%------------------------------------------------------------------------------%

\section{Research Experience in International Institutions}\label{sec:research}
\begin{adcvtabletwo}
  \positionheading
    {Jan--Jul 2022}
    {Visiting \phd Research Fellow}
    {Integrated Systems Laboratory -- ETH Zürich (Switzerland)}
    {\url{www.iis.ee.ethz.ch}}
  \adcvrowtwo{Supervisor: \lucabenini}{}
  \adcvrowmulti{Research topics included
    \begin{itemize*}[afterlabel= , label=]
      \item Energy-Aware Scheduling of Real-Time Tasks,
      \item Dynamic Voltage and Frequency Scaling (DVFS) on Embedded Platforms,
      \item System Monitoring, and
      \item Linux Kernel Development,
    \end{itemize*}
  }
\end{adcvtabletwo}

%------------------------------------------------------------------------------%

\section{Working Experience}\label{sec:experience}

\begin{adcvtabletwo}
  \positionheading
    {2013--2022}
    {IT Textbook Author}
    {Zanichelli editore S.p.A (Italy)}
    {\url{www.zanichelli.it}}
  %
  \adcvrowmulti{ I worked as a project collaborator and coauthor of several high
    school textbooks for IT classes. All of the chapters and books I authored
    focus on the essentials of mobile application development for Android OS,
    from the basics to more advanced use-case examples.
    %
    Refer to the \textbf{\hyperref[sec:books]{Books}} and
    \textbf{\hyperref[sec:bookchapters]{Book Chapters}} sections for the full
    list of textbooks I coauthored.
    }
  \adcvrowskip

  %----------------------------------------------------------------------------%

  \positionheading
    {2017--2018}
    {E-learning Technical Specialist}
    {\unipisa}
    {\url{www.unipi.it}}
  %
  \adcvrowmulti{To support my studies, I worked a couple of years as on-site
    support to technical and teaching personnel for the University for practical
    aspects of live streaming classes and conferences and the post-production of
    videos of an e-learning platform. Duties included managing recording
    equipment, streaming software, and providing hands-on support in case of
    live failures.}
  \adcvrowskip

  %----------------------------------------------------------------------------%

  \positionheading
    {2013--2018}
    {Tutor}
    {Independent Contractor, Livorno (Italy)}
    {}
  %
  \adcvrowmulti{Throughout my university years, I managed to help some high
    school students in STEM fields as their tutor, helping them with assignments
    in classes such as Computer Engineering, IT, Maths, Electronics, and
    Physics.}
  % \adcvrowskip

  %----------------------------------------------------------------------------%
\end{adcvtabletwo}

\section{Education}\label{sec:education}

\begin{adcvtabletwo}

% \begin{adcvtabletwo}
  \adcvrowtwo{\textbf{\mastersc} in Embedded Computing Systems}{2019}
  \adcvrowtwo{\unipi{} {\textbf{and}} \santannapisa}{}
  %
  % \ifextended
    \adcvrowmulti{
      Specialized curriculum in system programming, embedded and real-time
      systems, mechatronics, computer architectures and component frameworks.
      %
      Final grade: \textit{Summa cum laude}. Subject of the dissertation: {\em
      Design and Implementation of a Performance Testing Framework for
      High-Performance Inter-Container Communications}. Supervisor: {\em
      \tomcucinotta}.
    }
  % \fi
  \adcvrowskip

  %----------------------------------------------------------------------------%

  \adcvrowtwo{\textbf{\bachelorsc} in Computer Engineering}{2016}
  \adcvrowtwo{\unipisa}{}
  %
  % \ifextended
    \adcvrowmulti{
      Specialized curriculum in computer science and engineering, computer
      architectures, system programming, industrial automation and control
      systems.
      %
      Final grade: \textit{Summa cum laude}. Subject of the dissertation (in
      Italian): {\em Dynamic and Interactive Crisis Mapping and Generation}.
      Supervisor: {\em Prof.\ Marco Avvenuti}.
    }
  % \fi
  \adcvrowskip

  %----------------------------------------------------------------------------%

  \adcvrowtwo{\textbf{Secondary Diploma} as IT Professional}{2013}

  \adcvrowtwo{Istituto Tecnico Industriale Statale G.\ Galilei, Livorno (Italy)}{}
  % \ifextended
  \adcvrowtwo{\url{www.galileilivorno.gov.it}}{}
  % \fi
  %
  % \ifextended
    \adcvrowmulti{
      Curriculum in computer science, programming patterns, web technologies,
      mobile app development, networking, and system programming.
      %
      Final grade: \textit{100/100}.
    }
  % \fi

\end{adcvtabletwo}

\section{International Awards for Research Activities}\label{sec:awards}
\begin{adcvtabletwo}
  \adcvrowtwo{\textbf{Best Paper Award}}{2020} \adcvrowmulti{Gabriele Ara,
  Tommaso Cucinotta, Luca Abeni, Carlo Vitucci, for the work
  "\textit{\textbf{Comparative Evaluation of Kernel Bypass Mechanisms for
  High-performance Inter-Container Communications}}" at the 10th International
  Conference on Cloud Computing and Services Science (CLOSER), 2020}
\end{adcvtabletwo}

\section{Participation in Projects}\label{sec:projects}

\begin{adcvtabletwo}
\positionheadinglong
    {2020--present}
    {AMPERE}
    {A Model-driven development framework for highly Parallel and EneRgy-Efficient computation supporting multi-criteria optimisation}
    {\url{www.ampere-euproject.eu}}
    %
    \adcvrowmulti{Project funded by the European Union under grant agreement
      no.\ 871669. AMPERE envisages the development of novel methods and tools
      for constructing accurate models of proposed systems in computing
      platforms. Thus, system constraints are efficiently dealt with, while
      ensuring performance targets are met. AMPERE's computing software will
      also help improve overall system efficiency along with fulfilment of
      non-functional requirements. Contribution: I authored two publications
      related to the project. Personal research activities focus on the modeling
      of real-time tasks on embedded platform, real-time scheduling of complex
      task graphs, heterogeneous architectures, system monitoring, and kernel
      development. }

\end{adcvtabletwo}
%------------------------------------------------------------------------------%

\section{Teaching Experience}\label{sec:teaching}
\begin{adcvtabletwo}

  \positionheadinglong
    {2018--2019}
    {High School Teaching Professional}
    {Istituto di Istruzione Superiore ``Vespucci-Colombo'', Livorno (Italy)}
    {\url{vespucci.edu.it}}
    %
    \adcvrowmulti{For one academic year, I worked as a part-time professor (16~h
      per week) in an Italian high school, teaching IT and IT Laboratory to both
      \nth{10} and \nth{11}-grade students.}
    % \adcvrowskip

  %----------------------------------------------------------------------------%

  % \positionheadinglong
  %   {2020}
  %   {Teacher}
  %   {\santannapisa}
  %   {3~h}
  %   %
  %   \adcvrowskip

\end{adcvtabletwo}

\subsection{}{Supervision of \msc Students}
\label{sec:supervision}

\begin{adcvtabletwo}

  \adcvrowtwo{\textbf{Jacopo Malvatani}}{2021}
  % \ifextended
    \adcvrowtwo{Master student in Embedded Computing Systems}{}
    \adcvrowtwo{\santannapisa}{}
  % \else
  \adcvrowtwo{Master student in Embedded Computing Systems, \santannapisa}{}
  % \fi
  %
  \adcvrowmulti{Subject of the dissertation: {\em Enabling DPDK-based
  communications in MySQL for high-performance NFV applications}}
  \adcvrowskip

  \adcvrowtwo{\textbf{Leonardo Lai}}{2020}
  % \ifextended
    \adcvrowtwo{Master student in Embedded Computing Systems}{}
    \adcvrowtwo{\santannapisa}{}
  % \else
  \adcvrowtwo{Master student in Embedded Computing Systems, \santannapisa}{}
  % \fi
  %
  \adcvrowmulti{Subject of the dissertation: {\em Implementation and Evaluation
  of High-Performance Userspace Networking Mechanisms for Virtualized Network
  Functions}}

  %----------------------------------------------------------------------------%
\end{adcvtabletwo}


%------------------------------------------------------------------------------%

% \clearpage


%------------------------------------------------------------------------------%

% \ifextended
% \clearpage
% \else
% \fi


%------------------------------------------------------------------------------%
%------------------------------------------------------------------------------%

% \ifextended
% \else
% \clearpage
% \fi
% \section{Education}\label{sec:education}
% \end{adcvtabletwo}

%------------------------------------------------------------------------------%
%------------------------------------------------------------------------------%

\section{Professional Service as Reviewer}\label{sec:reviews}
\begin{adcvtabletwo}
  \adcvrowmultifake{\textbf{Secondary Reviewer for International Conferences}:
    ISORC 2020, ISORC 2021, RTCSA 2021, RTSS 2022.}
\end{adcvtabletwo}

\section{Certificates}\label{sec:certificates}
\begin{adcvtabletwo}
  \adcvrowtwo{\textbf{Professional Engineer Abilitation by the Italian Engineering Society}}{2020}
  \adcvrowskip

  \adcvrowtwo{\textbf{CCNA Exploration}: Accessing the WANs, Certificate of Course Completion}{2013}
  \adcvrowtwo{\textbf{CCNA Exploration}: LAN Switching and Wireless, Certificate of Course Completion}{}
  \adcvrowtwo{\textbf{CCNA Exploration}: Routing Protocols and Concepts, Certificate of Course Completion}{}
  \adcvrowtwo{\textbf{CCNA Exploration}: Network Fundamentals, Certificate of Course Completion}{}
  \adcvrowskip

  \adcvrowtwo{\textbf{Europen Computer Driving License (ECDL)}}{2012}
  \adcvrowskip

  \adcvrowtwo{\textbf{Trinity's Graded Examinations in Spoken English} -- Grade 7 (CEFR B2)}{2012}
  \adcvrowskip
\end{adcvtabletwo}

%------------------------------------------------------------------------------%

\section{Presentations and Seminars}\label{sec:presentations}

\begin{adcvtabletwo}
  \adcvrowtwo{``\textbf{Log4Shell: the dangers of hidden complexity and unsafe defaults}''}{2022}
  \adcvrowmulti{Seminar, Pisa, Oct 4, 2022}
  \adcvrowskip

  \adcvrowtwo{``\textbf{Simulating Execution Time and Power Consumption of Real-Time
    Tasks on Embedded Platforms}'' \href{https://www.youtube.com/watch?v=hdNFtkSmIG4}{[Online on YouTube]}}{2022}
    \adcvrowmulti{Presentation for SAC 2022 Conference, April 25, 2022}
  \adcvrowskip

  \adcvrowtwo{``\textbf{High-Performance Networking for Cloud and NFV Scenarios}''}{2021}
  \adcvrowmulti{Presentation of research activities for Ericsson (Sweden), May 28, 2021}
  \adcvrowskip

  \adcvrowtwo{``\textbf{Comparative Evaluation of Kernel Bypass Mechanisms for
    High-performance Inter-Container Communications}'' \href{https://www.youtube.com/watch?v=cQ3ecv6TVZc}{[Online on YouTube]}}{2020}
  \adcvrowmulti{Presentation for CLOSER 2020 Conference, Prague, Czech Republic, May 7, 2020}
  \adcvrowskip

  \adcvrowtwo{``\textbf{On the Use of Kernel Bypass Mechanisms for High-Performance
    Inter-Container Communications}''}{2019}
  \adcvrowmulti{Presentation for VHPC 2019 Workshop, Frankfurt, Germany, June 20, 2019}
  % \adcvrowskip
\end{adcvtabletwo}


%------------------------------------------------------------------------------%

\section{Dissemination Activities}\label{sec:dissemination}
\begin{adcvtabletwo}
  \adcvrowtwo{``\textbf{Effects of latency on real-time multimedia applications}''}{2021}
  \adcvrowtwo{Dissemination Activity (Italian)}{}
  \adcvrowtwo{Duration: 5~h}{}
  \adcvrowmulti{Demonstration as part of the BRIGHT initiative on behalf of \santannapisa.}
  \adcvrowskip

  \adcvrowtwo{``\textbf{Il (non) ruolo del Sistema Operativo nelle
    elaborazioni distribuite ad alte prestazioni}''}{2020}
  \adcvrowtwo{Orientation Seminar (Italian)}{}
  \adcvrowtwo{Duration: 2~h}{}
  \adcvrowmulti{Orientation to high school students for \santannapisa. The talk
    describes what is the role of having an OS in a modern machine and how this
    can impact the performance of critical systems in cloud scenarios.}

\end{adcvtabletwo}

%------------------------------------------------------------------------------%
%------------------------------------------------------------------------------%

% \ifextended
% \vspace{3em}
% \clearpage
% \else
% \ifpublist
% \clearpage
% \fi
% \fi
\section{Publications}\label{sec:publications}

\newcommand{\fullpublicationslist}{
  \begin{refsection}
    \nocite{Serra21JSA}
    \nocite{Ara20CCIS}
    \printbibliography[title={Peer-reviewed Journals\label{sec:journals}}, heading=subbibliography]
  \end{refsection}

  \clearpage

  \begin{refsection}
    \nocite{Ara2022Simulating}
    \nocite{Lai21IEEENFVSDN}
    \nocite{Serra20}
    \nocite{Ara20CLOSER}
    \nocite{Ara19}
    \printbibliography[title={Peer-reviewed Conference and Workshop Proceedings\label{sec:conferences}}, heading=subbibliography]
  \end{refsection}

  % \ifextended
  % \clearpage
  % \fi

  \begin{refsection}
    \nocite{MF2022}
    \printbibliography[title={Books\label{sec:books}}, heading=subbibliography]
  \end{refsection}

  \begin{refsection}
    \nocite{MC2020}
    \nocite{MF2017}
    \nocite{MC2015}
    \nocite{MF2014}
    \printbibliography[title={Book Chapters\label{sec:bookchapters}}, heading=subbibliography]
  \end{refsection}
}

% \ifextended
  \fullpublicationslist
% \else
  % \ifpublist
  %   \fullpublicationslist
  % % \else
  %   Search me on
  %   \href{htts://scholar.google.com/citations?user=nl1RmecAAAAJ}{Google Scholar}
  %   for the list of my peer-reviewed conference proceedings and journals. Or
  %   have a look at \href{https://gabrieleara.github.io/publications}{my
  %   publications page}.
  % \fi
% \fi


%------------------------------------------------------------------------------%
%------------------------------------------------------------------------------%


\section{Professional Skills}\label{sec:skills}

\newcommand*{\lang}[1]{{\linktext \textbf{#1}}\xspace}
% \multicolumn{2}{@{}p{\textwidth}}{\lighttext #1}\tabularnewline

\begin{adcvtabletwo}
  \adcvrowmultifake{\textbf{Communication skills} Comfortable interacting with
    people with cross-cultural backgrounds from around the world. Gave
    presentations to moderate international audiences for conferences and
    workshops. Can communicate strengths and weaknesses of his work.}
  \adcvrowskip

  %----------------------------------------------------------------------------%

  \adcvrowmultifake{\textbf{Organisational/Managerial skills} Can both work
    in small teams and manage to get things done by himself. Comfortable leading
    other people when necessary to organize the work. Excellent debugging and
    investigative skills for software/computer systems.}
  \adcvrowskip

  %----------------------------------------------------------------------------%

  \adcvrowskip
  \adcvrowtwo{\large \textbf{Job-related skills}}{}
  \adcvrowskip
  \adcvrowmulti{Following is the list of known programming languages and tools
  organized by proficiency level.}
  \adcvrowskip

  \adcvrowtwo{\textbf{Excellent knowledge and proficiency:}}{}
  \adcvrowmulti{
    \begin{itemize}
      \item \lang{C++}, including the understanding of language specification
        and internals, standard libraries, compilers optimization, build and
        packaging tools, up to the most recent C++ language standard
        specification (C++20).
      \item \lang{C}, including the understanding of language internals,
        standard libraries, POSIX, Linux-specific libraries, Linux system
        programming, microcontrollers programming.
    \end{itemize}
  }

  \adcvrowtwo{\textbf{Proficient knowledge:}}{}
  \adcvrowmulti{
    \begin{itemize}
      \item \lang{Virtualization} technologies, including Docker, LXC, LXD, and
      custom solutions based on {\em cgroups} and {\em namespaces};
      \item \lang{High-Performance Networking} frameworks, including DPDK, FD.io
      VPP, Open vSwitch;
      \item \lang{Linux} management and administrative tools for machine
        configuration and troubleshooting.
      \item \lang{Linux kernel} internals and hacking, module development,
        real-time schedulers internals;
      \item \lang{Unix}-based systems command line, scripting in \lang{Sh} and
        \lang{Bash};
      \item \lang{Java} (standard libraries, JavaFX, Spring, Apache and Servlet
        libraries);
      \item \lang{Android} App programming (using Java+XML);
    \end{itemize}
  }
  % \adcvrowskip

  \adcvrowtwo{\textbf{Good knowledge:}}{}
  \adcvrowmulti{
    \begin{itemize}
      \item \lang{Python} for scripting, data analysis, modeling, and visualization;
      \item \lang{Matlab} and Simulink for scripting, data analysis, and system
      control or simulation;
      \item \lang{Web} development using HTML5, PHP5, CSS3, XML, JSON, REST patterns;
      \item \lang{JavaScript} (standard libraries, JQuery, D3);
      \item \lang{SQL} and MySQL software suite;
      \item \lang{CISCO} technologies and CISCO IOS;
    \end{itemize}
  }
  % \adcvrowskip

  %----------------------------------------------------------------------------%
\end{adcvtabletwo}

%------------------------------------------------------------------------------%
%------------------------------------------------------------------------------%

\section{Languages}\label{sec:languages}

% \ifextended
  \begin{adcvlanguages}
    \adcvmothertongue{Italian}
    \adcvlanguagesheader
    \adcvlanguage{2}{English}{\adcvCTwo}{\adcvCTwo}{\adcvCTwo}{\adcvCOne}{\adcvCTwo}
    \adcvlanguagesfooter
    \adcvlanguagesfootnote{2}{2012 -- Trinity's Graded Examinations in Spoken English - Grade 7 (CEFR B2)}
  \end{adcvlanguages}
% \else
  % \textbf{Italian}: Native proficiency

  % \textbf{English}: Full professional proficiency
% \fi

%------------------------------------------------------------------------------%
%------------------------------------------------------------------------------%

\section{Personal Interests}\label{sec:interests}

Science, Technology, Traveling, Playing and Listening to Music, Food, Movies,
Trekking

% \ifextended

% \fi

\end{document}
